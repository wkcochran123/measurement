% ======================================================================
% INSTRUCTIONS WHEN EDITING THIS PROJECT
% ======================================================================
%
% 1. Do not guess labels or numbers.
%    Always use \ref{...} exactly as requested. Never invent labels.
%
% 2. When asked to insert a reference, write it literally as
%       Thought Experiment~\ref{te:lorentz}
%    or whichever label is given, without adding chapter, section, or numbers.
%
% 3. Use exact hierarchy:
%       \chapter{...}
%       \section{...}
%       \subsection{...}
%       \subsubsection{...}
%    Never invent section names or change levels.
%
% 4. Single, clean LaTeX block output only.
%    No prose, no explanation, no commentary outside the code block.
%
% 5. ASCII only in body text. No Unicode punctuation.
%
% 6. BibTeX citation keys are lastnameYYYY and must be sorted alphabetically
%    inside each citation command.
%
% 7. SOFT MARGINS OF 100 characters for all generated prose.
%
% 8. Never write explanations, apologies, or meta-text unless the user asks.
%    Do not self-justify, speculate, question instructions, or add commentary.
%    The output should assume the user already knows what they want and expects
%    exact compliance, not discussion.
%
% 9. Add labels to all equations, definitions, sections, chapters, and thought experiments:
%      \label{eq:[SHORT-NAME-WITH-DASHES]}
%      \label{def:[SHORT-NAME-WITH-DASHES]}
%      \label{se:[SHORT-NAME-WITH-DASHES]}
%      \label{chap:[SHORT-NAME-WITH-DASHES]}
%      \label{te:[SHORT-NAME-WITH-DASHES]}
%
%10. The theorem phenomenon is being replaced by phenom. When asked to write a phenomenon
%    phenom theorem.
%
%11. Do not gloss words
% ======================================================================
% END OF INSTRUCTIONS
% ======================================================================



\newtheorem{theorem}{Theorem}
\makeatletter
\def\@noftheorem#1#2{}
\makeatother
\newtheorem{axiom}{Axiom}
\newtheorem{remark}{Remark}
\newtheorem{definition}{Definition}
\newtheorem{proposition}{Proposition}
\newtheorem{corollary}{Corollary}
\newtheorem{equivalence}{Equivalence}
\newtheorem{law}{Law}
\newcommand{\Top}{\operatorname{Top}}
\newcommand{\J}{\mathbf{J}}        % Universe tensor
\newcommand{\U}{\mathbf{U}}        % Universe tensor
\newcommand{\T}{\mathbf{T}}        % Universe tensor
\newcommand{\E}{\mathbf{E}}        % Event tensor
\newcommand{\R}{\mathbf{R}}        % Event tensor
\newcommand{\V}{\mathcal{V}}       % Variation space
\newcommand{\Ledger}{\mathcal{L}}       % Ledger
\newcommand{\Refine}{\mathcal{R}}       % Ledger
\newcommand{\Rhat}{\hat{R}}       % Measurement space
\newcommand{\Talg}{\mathcal{T}(\V)}    % Tensor algebra
\newcommand{\Eset}{\mathcal{E}}    % Event set
\newcommand{\Part}{\mathsf{Part}}
% Entropy operator / functional
\newcommand{\Entropy}{\mathcal{S}}
\newcommand{\Blocks}[1]{\mathrm{Bl}(#1)}
\newcommand{\fold}{\mathop{\bigcirc}}


\newcommand{\citeph}[1]{%
  #1\footnote{See Phenomenon~\ref{ph:#1}}%
}


% --- House style for N.B. callouts (inline, bold, em-dash) ---
\newcommand{\NB}[1]{%
  \par\noindent\textbf{N.B.---}#1\hfill$\square$\par
  }


\newcommand{\addfiletotoc}[1]{%
\addcontentsline{toc}{section}{#1}%
}

% Proof (Sketch) environment
\newcommand{\psklabel}{}
\newenvironment{proofsketch}[1]{%
  \renewcommand{\psklabel}{#1}%
  \begin{proof}[Proof (Sketch)]%
}{%
  \end{proof}%
  \medskip
}



\newenvironment{coda}[1]{
  \section*{Coda: #1}
  \addcontentsline{toc}{section}{Coda: #1}
}{
}

\hyphenation{treat-ed}

% Length operator (number of folded factors)
\DeclareMathOperator{\len}{len}

% Common boundary notation and restriction to boundary
\newcommand{\Boundary}[2]{\partial(#1,#2)} % usage: \Boundary{\U^A}{\U^B}
\newcommand{\RestrictToBoundary}[2]{#1\!\upharpoonright_{\Boundary{#2}{#1}}}
% Example: \U^A\!\upharpoonright_{\partial(\U^A,\U^B)}

\makeatletter
\renewcommand{\l@chapter}{\@dottedtocline{0}{0em}{1.5em}}
\makeatother

% Usage: \PropositionSection{universe-tensor}
% Produces:
%   \section{Proposition~\ref{prop:universe-tensor}}
%   \label{app:universe-tensor}
%
\newcommand{\propproof}[1]{%
\section{Proposition~\ref{prop:#1}}%
\label{app:#1}%
}

\newtheorem{phenomenon}{Phenomenon (old)}
\newtheorem{phenomthm}{Phenomenon} % New structured phenomenon theorem


\makeatletter

\newif\ifph@statement
\newif\ifph@origin
\newif\ifph@observation
\newif\ifph@constraint
\newif\ifph@consequence
\newif\ifph@invariant
\newif\ifph@refinement

\newcommand{\ph@resetflags}{%
  \global\ph@statementfalse
  \global\ph@originfalse
  \global\ph@observationfalse
  \global\ph@constraintfalse
  \global\ph@consequencefalse
  \global\ph@invariantfalse
  \global\ph@refinementfalse
}

\newcommand{\ph@requireflags}{%
  \ifph@statement\else
    \PackageError{phenom}{Missing required block: Statement}{}%
  \fi
  \ifph@origin\else
    \PackageError{phenom}{Missing required block: Origin}{}%
  \fi
  \ifph@observation\else
    \PackageError{phenom}{Missing required block: Observation}{}%
  \fi
  \ifph@constraint\else
    \PackageError{phenom}{Missing required block: Operational Constraint}{}%
  \fi
  \ifph@consequence\else
    \PackageError{phenom}{Missing required block: Consequence}{}%
  \fi
}

\newenvironment{phenom}[1]%
{%
  \ph@resetflags
  \begin{phenomthm}[#1]%
  $\quad$

  \begin{list}{}{\leftmargin=1.5em \rightmargin=0pt \itemsep=0.5ex}%
}%
{%
  \end{list}%
  \ph@requireflags
  \end{phenomthm}%
  \medskip
}

\newcommand{\PhStatement}{\global\ph@statementtrue\item[\textbf{Statement.}]}
\newcommand{\PhOrigin}{\global\ph@origintrue\item[\textbf{Origin.}]}
\newcommand{\PhObservation}{\global\ph@observationtrue\item[\textbf{Observation.}]}
\newcommand{\PhConstraint}{\global\ph@constrainttrue\item[\textbf{Operational Constraint.}]}
\newcommand{\PhConsequence}{\global\ph@consequencetrue\item[\textbf{Consequence.}]}
\newcommand{\PhInvariant}{\global\ph@invarianttrue\item[\textbf{Invariant.}]}
\newcommand{\PhRefinement}{\global\ph@refinementtrue\item[\textbf{Refinement.}]}


\makeatother

