\chapter*{Preface}

Thank you for your interest in this work.  This manuscript is an ongoing attempt
to articulate a minimal theory of measurement grounded in discrete records,
enumeration, and refinement.  The aim is not to introduce new physical postulates,
but to make explicit which structures are forced by the act of recording itself,
and which are optional representational choices.

\paragraph{Overview and goals.}
The central goal of this work is to develop a theory of measurement in which
facts are treated as entries in an ordered ledger and all further structure is
introduced only when it can be recovered from refinement of that record.  The
theory is presented in two parallel forms.  First, the mathematical development
is given in prose, organized into chapters that introduce definitions,
constraints, and phenomena in a fixed logical order.  Second, the same structures
are formalized and verified in the Lean proof assistant, providing machine-checked
proofs of the core propositions.

\paragraph{State of the manuscript.}
The manuscript is under active development.
Chapter~1, which introduces ledgers, enumeration, and the collection of facts, is
currently in rough draft form but structurally complete.
Chapter~2, which introduces instruments as refinable models that generate and
coordinate records, has many of its major components in place, though portions
remain provisional.
Chapter~3 marks the beginning of less developed material and currently consists
of incomplete outlines and exploratory arguments.

\paragraph{State of the formalization.}
The Lean formalization mirrors the structure of the manuscript.
The definitions and propositions of Chapter~1 have been implemented and verified.
Formalization of Chapters~2 and~3 is underway, with core interfaces established
and additional constructions in progress.

\paragraph{Goal of the proof.}
A guiding objective of this project is to clarify the role of the Continuum
Hypothesis as a representational choice rather than a statement about physical
reality.  In particular, we aim to show that assuming the Continuum Hypothesis
corresponds to modeling carrier particles as having indefinitely dissectable
volume, while assuming its negation corresponds to modeling carriers as atomic
and volume-free.  These two assumptions give rise to distinct representational
models that nevertheless span the same space of communicable simulations.  Within
the ledger framework, no measurement can be communicated that expresses the volume
of such carriers; only counts and correlations are admissible.

\paragraph{Repository and versioning.}
All source files for the manuscript and the Lean formalization are maintained in
the public repository
\begin{center}
\texttt{https://github.com/wkcochran123/measurement}
\end{center}
The current working release is version~0.1.1.

\paragraph{Request for sponsorship.}
This manuscript is intended for submission to the arXiv under the category
\texttt{cs.IT}.  Readers who are eligible and willing to sponsor the submission
are kindly invited to do so using the code \texttt{C9GIVH}.

