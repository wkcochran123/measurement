\chapter*{Abstract}
\addcontentsline{toc}{section}{Abstract}

This work develops a finite, instrument-based approximation to space--time
structure by analyzing two distinct linearizations associated with a single
invariant field: the Gateaux approximation and the Frechet approximation.  Each
linearization is realized through a finite computational device.  The Gateaux
approximation is implemented via a Newton-style search, while the Frechet
approximation is implemented via a bisection-based search, reflecting their
respective sensitivities to local direction and global constraint.

To support iteration, explicit maps from the integers to representative
continua are constructed.  For Newton search, convergence is represented by
Cauchy sequences; for bisection search, completion is represented by Cantor's
construction.  These representations are treated as idealized instruments,
distinct from the realized devices that execute finite refinement and commit
discrete records.  The framework maintains a strict separation between
representational completion and operational certification.

Within this setting, the residual structure arising from each approximation is
computed.  The Gateaux approximation leaves a directional residue associated
with anticipatory refinement, while the Frechet approximation leaves a
constraint-based residue associated with global consistency.  The interesection of
these residues characterizes the full space of admissible continuations under
finite measurement.  This space is identified with the union of all possible
physical laws compatible with the instrument, understood not as ontological
necessities but as stable invariants that survive finite refinement under
recoverability constraints.

