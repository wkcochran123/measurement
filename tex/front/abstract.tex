\chapter*{Abstract}
\addcontentsline{toc}{section}{Abstract}


We propose a foundational framework for physical measurement that does not assume
a pre-existing continuum, but instead begins from the experimental ledger: a
finite, growing record of distinguishable events. Guided by six axioms drawing
from Peano, Kolmogorov, and Planck, observation is formalized as an irreversible
process of refinement, in which new distinctions are appended to a history that
cannot be erased.

In the first part of the work, we construct the Causal Universe Tensor, an
algebraic object that encodes admissible event histories and their causal
relations. Familiar continuous structures, including time and geometry, are
shown to arise not as primitive elements, but as smooth representational
surrogates required to interpolate intervals of verified silence between
discrete records.

The latter part reframes physical dynamics in this setting. Physical laws appear
as bookkeeping constraints that preserve consistency across observers, rather
than as generators of motion in a prior spacetime. Within this framework, the
monotonicity of causal entropy ($\Delta S \geq 0$) is not an additional law, but a
structural consequence: because the ledger is append-only, the number of
distinguishable events can only increase.


