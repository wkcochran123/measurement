\chapter{Instruments}
\label{ch:instruments}

Measurement does not begin with records or histories, but with instruments. An
instrument specifies the distinctions an observer is capable of making and the
expectations under which those distinctions are produced. Before a ledger may
be formed or refinement discussed, the instrument itself must be defined as a
static object, independent of time or accumulation. Without an instrument,
there is nothing that can be said to have been measured, recorded, or compared.

An instrument encodes the current understanding of a phenomenon. It reflects
what distinctions are believed to matter and which variations are to be treated
as irrelevant. This understanding may be incomplete or even incorrect, but it
is always explicit in the structure of the instrument. The instrument therefore
represents a commitment: it declares in advance what counts as an observable
difference.

In this sense, an instrument is deterministic. If the history of the world were
replayed exactly as before, the instrument would produce the same reading. The
phrase ``exactly as before'' is operational rather than metaphysical. It refers
to any initial orientation or configuration of the instrument that yields the
same response when presented with the same triggering conditions. Determinism
here is a property of construction, not of the underlying phenomenon.

For the purposes of this work, an instrument is composed of two conceptual
components: a sensor and a reading. The sensor is the part of the instrument
that physically interacts with the phenomenon. It is constructed using
well-established engineering practices and calibrated against known standards.
The reading is the symbolic output presented to the observer, drawn from a
finite and well-defined set of possible indications.

The separation between sensor and reading is not merely descriptive. It encodes
a causal ordering. The sensor is triggered first, responding to the phenomenon,
and only afterward is a reading produced. This ordering is irreversible: a
reading cannot occur without a prior sensor interaction. In this way, the
instrument itself embodies an arrow of time, even before any notion of history
or record is introduced.

Returning to the speedometer in an automobile, the sensor measures rotational motion
of the drivetrain, while the reading displays a discrete marking on a dial or a
digital display. If the same physical motion occurs again under the same
conditions, the speedometer produces the same reading. The instrument does not
store past speeds; it merely reacts according to its design.

A radar gun illustrates the same structure more explicitly. The sensor emits a
signal and receives a reflected response, while the reading reports a discrete
speed value. The emission and reception occur before the display is updated.
Even though the underlying physical interaction involves continuous quantities,
the instrument reports only a finite symbol. The arrow of time is enforced by
the sequence sensor--then--reading, not by any assumption about temporal
continuity.

In both cases, the instrument is built with a model in mind. The speedometer
assumes a relationship between rotation and linear velocity. The radar gun
assumes a relationship between frequency shift and relative motion. These models
are not inferred at the moment of measurement; they are compiled into the
instrument. The instrument therefore carries a predictor implicitly, even when
no explicit prediction is displayed.

This predictor need not be probabilistic. It may be a fixed rule determined by
geometry, mechanics, or electronics. What matters is that the instrument is
designed to map admissible physical interactions to symbols in a consistent way.
The predictor expresses what the instrument expects to observe next, given its
construction and calibration.

In this chapter we formalize instruments as symbolic and predictive structures.
We do not assume that the predictor is correct, optimal, or even well-matched to
the phenomenon. Those questions belong to later chapters. Here we require only
that the instrument specifies a finite set of symbols and a rule for producing
one of those symbols in response to interaction.

The purpose of this chapter is therefore foundational. By identifying the
minimal structure required for an instrument to exist, we isolate what must be
in place before measurement can proceed at all. Ledgers, refinement, causality,
and probability will be introduced later. They all depend on instruments, but
instruments depend only on distinguishability and construction.


\section{Symbol Alphabets}
\label{sec:alphabets}

Every instrument communicates through a finite set of distinguishable symbols.
These symbols represent the total expressive capacity of the instrument: no
measurement performed by the instrument may produce information outside this
alphabet.  Whatever structure the underlying phenomenon may possess, the
instrument can report only distinctions drawn from this fixed symbolic set.
Any claim that cannot be expressed in the instrument’s alphabet is, by
definition, unmeasurable by that instrument.

The restriction to a finite alphabet is not a limitation imposed for
convenience, but a structural requirement.  An instrument that could emit an
unbounded or continuously varying symbol would be incapable of calibration,
comparison, or repetition.  Finiteness ensures that every possible output of the
instrument can, in principle, be listed, indexed, and exhaustively compared
against past outputs.  It is this finiteness that makes measurement stable under
repetition.

A symbol alphabet enforces a discrete notion of distinguishability.  Two outputs
are either the same symbol or they are not; there is no intermediate symbolic
state.  Any apparent continuity in measurement arises from refinement of the
instrument or reinterpretation of its outputs, not from the alphabet itself.
By fixing the alphabet in advance, the instrument rules out the introduction of
unrecordable intermediate structure.

Formally, a symbol alphabet is a finite, totally ordered, enumerable set.  The
total order provides a canonical way to compare symbols, allowing one to speak
of one reading being greater than, less than, or equal to another.  This order
is intrinsic to the alphabet and does not depend on time, frequency of use, or
any statistical interpretation.

Enumerability supplies the mechanism by which symbols may be indexed.  Each
symbol is assigned a unique ordinal position within the alphabet, enabling
random access, iteration, and successor operations.  These operations are
purely structural and do not presuppose any notion of temporal sequence or
measurement history.  They are properties of the alphabet as a combinatorial
object.

The combination of finiteness, ordering, and enumerability ensures that symbol
alphabets may be manipulated without ambiguity.  Symbols may be stored,
compared, transmitted, and reconstructed exactly.  This is essential for later
constructions in which records and ledgers will depend on symbol indices rather
than on symbolic labels themselves.

It is important to emphasize that symbol alphabets carry no probabilistic or
temporal meaning on their own.  A symbol does not represent an event having
occurred at a particular time, nor does it encode likelihood or frequency.  Such
interpretations arise only after symbols are placed into records and histories.
At this stage, the alphabet is simply the grammar of what the instrument can
say.

Different instruments may employ different symbol alphabets even when designed
to interact with the same phenomenon.  One instrument may report coarse-grained
symbols, while another reports finer distinctions.  Neither alphabet is more
fundamental than the other; each reflects a design choice about which
distinctions are to be made explicit.

By isolating symbol alphabets as independent, static objects, we separate the
problem of representation from the problems of prediction and refinement.  The
alphabet fixes the space of possible readings in advance.  How those readings
are produced, ordered in time, or updated in light of new information will be
addressed in subsequent sections.


\section{Predictors}
\label{sec:predictors}

An instrument is not defined solely by what it can say, but also by what it
expects to say next.  This expectation is captured by a predictor: a function
that maps admissible contextual information to a symbol in the instrument's
alphabet.  The predictor represents the internal logic or model under which the
instrument operates.

Importantly, the predictor need not be correct.  It need only be compatible with
the instrument's alphabet.  Predictors are not empirical claims about the world,
but structural components of the instrument itself.  Their existence reflects
the fact that instruments are constructed with specific behaviors and responses
in mind.

\section{Instruments}
\label{sec:instruments}

An instrument is defined as a pair consisting of a finite symbol alphabet and a
predictor compatible with that alphabet.  This definition is intentionally
minimal: it includes exactly the structure required for an instrument to produce
a sequence of symbols and nothing more.

At this stage, instruments are static objects.  They do not evolve, they do not
refine, and they do not accumulate records.  The instrument merely specifies
what symbols may be produced and how the next symbol is anticipated, independent
of any ledger or history.

\section{Phenomena and Models}
\label{sec:phenomena}

Instruments are constructed to interact with phenomena.  A phenomenon specifies
the constraints under which events may occur and measurements may be taken.  It
encodes the admissible behaviors of the system being measured, without selecting
which behavior is realized.

Each phenomenon carries a model.  This model need not be complete or correct, but
it provides a structured description against which instruments are designed.
The model associated with a phenomenon determines which predictors are
meaningful and which instruments may be said to measure the phenomenon at all.

\section{Modeled and Bayesian Instruments}
\label{sec:instrument_types}

Instruments may be classified by the origin of their predictors.  A modeled
instrument has a predictor fixed by construction, derived directly from a
physical or geometric model embedded in the instrument's design.  Such
instruments implement their expectations rather than infer them.

A Bayesian instrument, by contrast, updates its predictor through inference over
admissible possibilities.  Its expectations are revised in light of new
information, subject to consistency with the underlying phenomenon.  This
distinction concerns only the source of the predictor and does not yet invoke
records, probabilities, or refinement.

\section{Compatibility}
\label{sec:compatibility}

An instrument is compatible with a phenomenon if its predictor respects the
constraints imposed by the phenomenon's model.  Compatibility does not require
accuracy, only admissibility: the instrument must not predict symbols that are
inconsistent with the phenomenon it is intended to measure.

Different instruments may be compatible with the same phenomenon while employing
different alphabets or predictors.  This multiplicity reflects the fact that
measurement depends on representational choices rather than unique physical
descriptions.  Questions of comparison and consistency are deferred until
ledgers and refinement are introduced.





